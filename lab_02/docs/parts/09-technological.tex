\chapter{Технологическая часть}

\section{Средства реализации}

Для программной реализации шифровальной машины был выбран язык C++ \cite{cpp}.
В данном языке есть все требующиеся инструменты для данной лабораторной работы.
В качестве среды разработки была выбрана среда Visual Studio 2022 \cite{clion}.




\section{Реализация алгоритма}

\begin{lstlisting}[label=lst:process-block,caption=Реализация алгоритма DES]
	bitset<64> process_block(bitset<64> value, bitset<64> key, bool decrypte = false)
	{
		// generate keys
		auto keys = generate_keys(key, decrypte);
		
		// initial permutation
		auto round_val = IP_f(value);
		
		// 16 rounds
		for (auto rkey : keys) {
			round_val = wround(round_val, rkey);
		}
		
		// final permutation
		auto final_val = FP_f(round_val);
		
		return final_val;
	}
\end{lstlisting}


\begin{lstlisting}[label=lst:ecb,caption=Реализация метода шифрования и дешифрования DES в режиме CBС]
vector<char> cypher(vector<char> input, vector<char> key, vector<char> vi)
{
	vector<char> buffer = {};
	vector<char> result = {};
	
	int last_cnt = 0;
	
	auto vi_initial = vchar_to_bitset64(vi);
	auto key_b = vchar_to_bitset64(key);
	
	
	for (auto sym : input) {
		if (buffer.size() < 8) {
			buffer.push_back(sym);
		}
		
		if (buffer.size() == 8) {
			// buffer size is 8 -> can be cyphered
			auto buf_b = vchar_to_bitset64(buffer) ^ vi_initial;
			
			auto tmp_b_1 = _des.process_block(buf_b, key_b);
			
			vi_initial = tmp_b_1;
			
			auto tmp_res = bitset64_to_vchar(tmp_b_1);
			
			result.insert(result.end(), tmp_res.begin(), tmp_res.end());
			
			buffer.clear();
		}
	}
	
	if (buffer.size() > 0 && buffer.size() < 8) {
		while (buffer.size() < 8) {
			buffer.push_back((char)0);
			last_cnt += 1;
		}
		auto buf_b = vchar_to_bitset64(buffer) ^ vi_initial;
		
		auto tmp_b_1 = _des.process_block(buf_b, key_b);
		
		vi_initial = tmp_b_1;
		
		auto tmp_res = bitset64_to_vchar(tmp_b_1);
		
		result.insert(result.end(), tmp_res.begin(), tmp_res.end());
	}
	
	result.push_back((char)last_cnt);
	
	
	return result;
}
}
\end{lstlisting}

\section{Тестирование}
\begin{table}[ht!]
	\begin{center}
		\captionsetup{justification=raggedright,singlelinecheck=off}
		\caption{\label{tbl:functional_test} Функциональные тесты}
		\begin{tabular}{|c|c|c|}
			\hline
			Входная строка & Выходная строка \\ 
			\hline
			HelloEnigma123 & >hfeinduehfo2m \\
			hello  & 1ma8r\\
			A & k\\
			<<>>  & <<>>\\
			\hline
		\end{tabular}
	\end{center}
\end{table}
\clearpage

\section*{Вывод}

В данном разделе были рассмотрены средства реализации, а также представлены листинги реализации шифровального алгоритма DES и режима работы CBC.