\chapter{Технологическая часть}

В данном разделе будут рассмотрены средства реализации, а также представлены листинги реализаций алгоритма шифрования машины <<Энигма>>.

\section{Средства реализации}
В данной работе для реализации был выбран язык программирования $C++$. Данный язык удоволетворяет поставленным критериям по средствам реализации.

\section{Реализация алгоритма}

В листингах \ref{lst:enigma1} представлена реализация алгоритма шифрования машины <<Энигма>>.

\begin{center}
    \captionsetup{justification=raggedright,singlelinecheck=off}
    \begin{lstlisting}[label=lst:enigma1,caption=Реализация алгоритма шифрования машины <<Энигма>>]
#include <iostream>

#include <iostream>
#include <vector>
#include <fstream>
#include <string>
#include <unordered_map>
using namespace std;

const int ALPHABET_SIZE = 64;

const string ALPHABET = "ABCDEFGHIJKLMNOPQRSTUVWXYZabcdefghijklmnopqrstuvwxyz0123456789.,";

class Rotor {
	public:
	Rotor(const string wiring, int position = 0)
	{
		string part1 = wiring.substr(0, position);
		string part2 = wiring.substr(position, ALPHABET_SIZE - position);
		wiring_ = part2 + part1;
		//wiring_ = wiring;
	}
	
	char encryptForward(char c) {
		int index = wiring_.find(c);
		return ALPHABET[index];
	}
	
	char encryptBackward(char c) {
		size_t index = ALPHABET.find(c);
		return wiring_[index];
	}
	
	void rotate() {
		char tmp = wiring_[0];
		for (int i = 0; i < ALPHABET_SIZE - 1; i++)
		{
			wiring_[i] = wiring_[i + 1];
		}
		wiring_[ALPHABET_SIZE - 1] = tmp;
	}
	
	private:
	string wiring_;
};

class Reflector {
	public:
	Reflector(const string wiring)
	{
		wiring_ = wiring;
		for (int i = 0; i < wiring.size(); i += 1)
		{
			if (i % 2 == 0)
			{
				wiring_[i] = wiring[i + 1];
			}
			else
			{
				wiring_[i] = wiring[i - 1];
			}
		}
	}
	
	char reflect(char c) {
		return wiring_[ALPHABET.find(c)];
	}
	
	private:
	string wiring_;
};

class Enigma {
	public:
	Enigma(const vector<Rotor>& rotors, const Reflector& reflector)
	: rotors_(rotors), reflector_(reflector) {}
	
	string encryptMessage(const string& message) {
		string encryptedMessage;
		for (char c : message) {
			if (isInAlphabet(c)) {
				encryptedMessage += encryptChar(c);
				rotateRotors();
			}
			else {
				encryptedMessage += c; 
			}
		}
		return encryptedMessage;
	}
	
	void encryptFile(const string& inputFile, const string& outputFile) {
		ifstream inFile(inputFile, ios::binary);
		ofstream outFile(outputFile, ios::binary);
		
		if (!inFile.is_open() || !outFile.is_open()) {
			cerr << "Error" << endl;
			return;
		}
		
		char buffer;
		while (inFile.get(buffer)) {
			if (isInAlphabet(buffer)) {
				buffer = encryptChar(buffer);
				rotateRotors();
			}
			outFile.put(buffer);
		}
		
		inFile.close();
		outFile.close();
	}
	
	private:
	vector<Rotor> rotors_;
	Reflector reflector_;
	int cnt_ = 0;
	

	char encryptChar(char c) {
		for (Rotor& rotor : rotors_) {
			c = rotor.encryptForward(c);
		}
		
		c = reflector_.reflect(c);
		
		for (auto it = rotors_.rbegin(); it != rotors_.rend(); ++it) {
			c = it->encryptBackward(c);
		}
		return c;
	}
	
	void rotateRotors() {
		cnt_++;
		rotors_[0].rotate();
		if (cnt_ % ALPHABET_SIZE == 0)
		rotors_[1].rotate();
		if (cnt_ % (ALPHABET_SIZE * ALPHABET_SIZE) == 0)
		rotors_[2].rotate();
		
	}
	
	bool isInAlphabet(char c) {
		return ALPHABET.find(c) != string::npos;
	}
};

\end{lstlisting}
\end{center}

\section{Тестирование}
\begin{table}[ht!]
	\begin{center}
		\captionsetup{justification=raggedright,singlelinecheck=off}
		\caption{\label{tbl:functional_test} Функциональные тесты}
		\begin{tabular}{|c|c|c|}
			\hline
			Входная строка & Выходная строка \\ 
			\hline
			HelloEnigma123 & >hfeinduehfo2m \\
			hello  & 1ma8r\\
			A & k\\
			<<>>  & <<>>\\
			\hline
		\end{tabular}
	\end{center}
\end{table}

\section*{Вывод}

Были представлены листинги реализаций алгоритма шифрования в машине <<Энигма>> согласно алгоритму, представленному в первой части, а также проведено тестирование разработанной программы.