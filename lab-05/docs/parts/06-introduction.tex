\chapter*{ВВЕДЕНИЕ}
\addcontentsline{toc}{chapter}{ВВЕДЕНИЕ}

Сжатие данных --- обратимое преобразование данных, производиме с целью уменьшения занимаемого ими объёма с целью уменьшения объёма данных, который требуется для хранения или передачи информации.

\textbf{Целью данной работы} является реализация в виде программы алгоритма сжатия данных Хафмана, обеспечить сжатие и разжатие произвольного файла с использованием разработанной программы, расчитывать коэффициент сжатия. Предусмотреть работу с пустым, однобайтовым файлами.

Для достижения поставленной цели необходимо выполнить следующие задачи:
\begin{enumerate}[label=\arabic*)]
	\item изучить алгоритм сжатия Хафмана;
	\item реализовать алгоритм сжатия Хафмана в виде программы, обеспечив возможности сжатия и разжатия произвольнго файла и расчёт коэффициента сжатия;
	\item протестировать разработанную программу, показать, что удаётся сжимать и разжимать файлы разных форматов;
	\item описать и обосновать полученные результаты в отчёте о выполненной лабораторной работе.
\end{enumerate}
