\chapter{Аналитическая часть}

\section{Алгоритм Хаффмана}

Алгоритм Хаффмана --- алгоритм сжатия данных, который формирует основную идею сжатия файлов. 
Кодирование Хаффмана --- это тип кодирования с переменной длиной слова
Был разработан в 1952 году аспирантом Массачусетского технологического института Дэвидом Хаффманом при написании им курсовой работы. 
В настоящее время используется во многих программах сжатия данных.

Основные этапы алгоритма сжатия с помощью кодов Хаффмана:

Сбор статистической информации для последующего построения таблиц кодов переменной длины

Построение кодов переменной длины на основании собранной статистической информации

Кодирование (сжатие) данных с использованием построенных кодов

Алгоритм состоит из следующих шагов:
\begin{itemize}
	\item сортировка выходных символов, не меняя местоположения символа, по вероятности их встречаемости в убывающем порядке;
	\item объединение двух символов с наименьшимс вероятностями в композицию символов с вероятностью, равной сумме исходных вероятностей;
	\item повторения предыдущего шага до тех пор, пока не получится композиция с вероятностью 1, которая называетя корнем. Полученная структрура называется деревом Хаффмана;
	\item проход по дереву от корня до соответсвующего символа и присвоение 0 (1) --- левой и 1 (0) --- правой ветви. 
\end{itemize}



