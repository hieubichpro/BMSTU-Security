\chapter{Технологическая часть}

\section{Средства реализации}

Для программной реализации шифровальной машины был выбран язык C++ \cite{cpp}.
В данном языке есть все требующиеся инструменты для данной лабораторной работы.
В качестве среды разработки была выбрана среда Visual Studio Code.

\section{Реализация алгоритма}

\begin{lstlisting}[caption=Алгоритм сжатия файла]
	void decompress(const char* filename, const ll Filesize)
	{
		const std::string fl = filename;
		FILE* iptr = fopen(std::string("../data/" + fl).c_str(), "rb");
		FILE* optr = fopen(std::string("../data/d_" + fl).c_str(), "wb");
		
		if (iptr == NULL)
		{
			perror("Error: File not found");
			exit(-1);
		}
		
		auto [padding, headersize] = decode_header(iptr);
		store_huffman_value();
		print_tree();
		
		byte ch;
		char counter = 7;
		ll size = 0;
		const ll filesize = Filesize - headersize;
		Node* traverse = root;
		ch = fgetc(iptr);
		while (size != filesize)
		{
			while (counter >= 0)
			{
				traverse = ch & (1 << counter) ? traverse->right : traverse->left;
				--counter;
				if (!traverse->left && !traverse->right) {
					fputc(traverse->character, optr);
					if (size == filesize - 1 && padding == counter + 1) {
						break;
					}
					traverse = root;
				}
			}
			++size;
			counter = 7;
			ch = fgetc(iptr);
		}
		fclose(iptr);
		fclose(optr);
	}
\end{lstlisting}

\begin{lstlisting}[caption=Алгоритм разжатия файла]
void decompress(const char* filename, const ll Filesize)
{
	const std::string fl = filename;
	FILE* iptr = fopen(std::string("../data/" + fl).c_str(), "rb");
	FILE* optr = fopen(std::string("../data/d_" + fl).c_str(), "wb");
	
	if (iptr == NULL)
	{
		perror("Error: File not found");
		exit(-1);
	}
	
	auto [padding, headersize] = decode_header(iptr);
	store_huffman_value();
	print_tree();
	
	byte ch;
	char counter = 7;
	ll size = 0;
	const ll filesize = Filesize - headersize;
	Node* traverse = root;
	ch = fgetc(iptr);
	while (size != filesize)
	{
		while (counter >= 0)
		{
			traverse = ch & (1 << counter) ? traverse->right : traverse->left;
			--counter;
			if (!traverse->left && !traverse->right) {
				fputc(traverse->character, optr);
				if (size == filesize - 1 && padding == counter + 1) {
					break;
				}
				traverse = root;
			}
		}
		++size;
		counter = 7;
		ch = fgetc(iptr);
	}
	fclose(iptr);
	fclose(optr);
}
\end{lstlisting}

\section{Тестирование}
Все тесты с файлами успешно пройдены.